
\begin{Large}
\begin{center}
\textbf{Trabajo Encargado} \\
\end{center}
\end{Large}

\section{Objetivos} 


\begin{itemize}

Realizar la Instalación de un servidor de Base de Datos
Oracle sobre un sistema operativo Linux.\\\\
Realizar la instalación de este servidor dentro de un ambiente virtualizado.\\

\end{itemize} 

\section{Requerimientos} 

\begin{itemize}
Conocimientos\\\\
Para el desarrollo de esta práctica se requerirá de los siguientes conocimientos básicos:\\
- Conocimientos básicos de comandos Linux a nivel de consola o terminal de texto.\\
- Conocimientos básicos de redes locales.\\\\

Hardware\\\\
Se necesitará un PC (computadora personal), con las siguientes características:\\
- 01 procesador de doble núcleo o superior\\
- 4Gb de memoria física (RAM) o superior\\
- Disco duro con 100Gb de capacidad y al menos 30Gb de espacio libre\\
- Unidad Lector de DVD.\\
- Interfaz de Red Ethernet activa.\\\\

Software\\\\
Asimismo se necesita los siguientes aplicativos:\\
- Sistema Operativo Windows XP o superior preinstalado, de preferencia actualizado    con todos los parches de seguridad.\\
- Instalador de Oracle Linux Release 5 (En DVD o archivo de tipo imagen .ISO).\\
- Instalador de Oracle Database 11g R2 (En DVD o archivo de tipo imagen .ISO).\\
- Instalador de Virtualbox versión 4.x o superior.\\

\end{itemize} 

\section{Pasos a seguir}
\begin{itemize}
    
    
\end{itemize}
