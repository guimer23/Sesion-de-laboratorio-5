\section{Cuestionario} 
	\item ¿Qué sucede al ejecutar los siguientes comandos?\\.
	  -STARTUP OPEN:\\
    La base de datos está completamente funcional. Para ello se abren los archivos de datos y los Redo Log y se comprueba la consistencia de los datos.\\
    
    -STARTUP MOUNT:\\
    Al estado anterior se añade la lectura de los archivos de control que permiten determinar cómo se ha de preparar la instancia. 
    Se buscan los archivos de datos y los Redo Log, comprobando su existencia en las rutas marcadas por el archivo de control.
    En este estado podemos conectar (como administradores) y realizar tareas como:\\
      -Cambio del nombre de los archivos de datos.\\
      -Activar el modo ARCHIVELOG.\\
      -Recuperación de la base de datos.\\
      -En definitiva, tareas sobre los archivos de la base de datos ya que aun no se han abierto sus datos.\\
      
	\begin{center}
	\includegraphics[width=14cm]{./Imagenes/imagen62} 
	\end{center}
  
    -STARTUP NOMOUNT:\\
    La instancia de base de datos está latente en memoria, con los procesos comunes funcionando. 
    Se abre el archivo de parámetros, se asigna en memoria el espacio para la SGA, 
    se lanzan los procesos en segundo plano, se abren los archivos de traza y alerta.\\
    
    -STARTUP FORCE:\\
    Hace SHUTDOWN ABORT y arranca la BD.\\
    
    -STARTUP RESTRICT:\\
    Es un modo especial de trabajo en el que la base de datos está abierta, pero solo se permite el acceso a usuarios 
    con permiso RESTRICTED (lo poseen los administradores) para hacer tareas especiales de administración.\\
    
    -STARTUP RECOVER:\\
    Especifica que la recuperación de medios se debe realizar, si es necesario, antes de iniciar la instancia. STARTUP RECOVER tiene el mismo efecto que emitir el comando RECOVER DATABASE e iniciar una instancia. Solo la recuperación completa es posible con la RECOVER option.\\
    
    -SHUTDOWN NORMAL:\\
    Espera a que terminen todas las transacciones en curso y todas las sesiones, 
    fuerza un checkpoint, además de cerrar todos los ficheros y destruir (parar) la instancia.\\
    
    -SHUTDOWN TRANSACTIONAL:\\
    Sólo espera a que terminen las transacciones en curso, fuerza un checkpoint, cierra los ficheros y destruye (para) la instancia.\\
    
    -SHUTDOWN ABORT:\\
     Cierra la instancia (destruye procesos background y SGA) sin esperar a desmontar ni cerrar la BD (como en una “caída”, ni hace checkpoint ni cierra ficheros). Requiere recovery de la instancia al arrancar (lo hace automáticamente el proceso SMON).\\

    -SHUTDOWN INMEDIATE:\\
    Hace rollback de todas las transacciones en curso y cierra todas las sesiones; cierra y desmonta la BD, además de forzar un checkpoint, cerrar ficheros y parar la instancia (como los anteriores).\\
    
   \item En el script lab_02_01.sql, se establecen privilegios de sistema, enliste los privilegios de sistema (DDL) 
   utilizados y describa cada uno de ellos.\\
   
   \item Enliste y describa los tipos de TableSpace que existen en Oracle.\\
   TIPOS DE TableSpace:\\
   - Tablespace SYSTEM.-\\
   - Tablespaces temporales.-\\
